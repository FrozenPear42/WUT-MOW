% Options for packages loaded elsewhere
\PassOptionsToPackage{unicode}{hyperref}
\PassOptionsToPackage{hyphens}{url}
%
\documentclass[
]{article}
\usepackage{lmodern}
\usepackage{amssymb,amsmath}
\usepackage{ifxetex,ifluatex}
\ifnum 0\ifxetex 1\fi\ifluatex 1\fi=0 % if pdftex
  \usepackage[T1]{fontenc}
  \usepackage[utf8]{inputenc}
  \usepackage{textcomp} % provide euro and other symbols
\else % if luatex or xetex
  \usepackage{unicode-math}
  \defaultfontfeatures{Scale=MatchLowercase}
  \defaultfontfeatures[\rmfamily]{Ligatures=TeX,Scale=1}
\fi
% Use upquote if available, for straight quotes in verbatim environments
\IfFileExists{upquote.sty}{\usepackage{upquote}}{}
\IfFileExists{microtype.sty}{% use microtype if available
  \usepackage[]{microtype}
  \UseMicrotypeSet[protrusion]{basicmath} % disable protrusion for tt fonts
}{}
\makeatletter
\@ifundefined{KOMAClassName}{% if non-KOMA class
  \IfFileExists{parskip.sty}{%
    \usepackage{parskip}
  }{% else
    \setlength{\parindent}{0pt}
    \setlength{\parskip}{6pt plus 2pt minus 1pt}}
}{% if KOMA class
  \KOMAoptions{parskip=half}}
\makeatother
\usepackage{xcolor}
\IfFileExists{xurl.sty}{\usepackage{xurl}}{} % add URL line breaks if available
\IfFileExists{bookmark.sty}{\usepackage{bookmark}}{\usepackage{hyperref}}
\hypersetup{
  pdftitle={Metody odkrywania wiedzy},
  pdfauthor={Rafał Galczak, Wojciech Gruszka},
  hidelinks,
  pdfcreator={LaTeX via pandoc}}
\urlstyle{same} % disable monospaced font for URLs
\usepackage[margin=1in]{geometry}
\usepackage{color}
\usepackage{fancyvrb}
\newcommand{\VerbBar}{|}
\newcommand{\VERB}{\Verb[commandchars=\\\{\}]}
\DefineVerbatimEnvironment{Highlighting}{Verbatim}{commandchars=\\\{\}}
% Add ',fontsize=\small' for more characters per line
\usepackage{framed}
\definecolor{shadecolor}{RGB}{248,248,248}
\newenvironment{Shaded}{\begin{snugshade}}{\end{snugshade}}
\newcommand{\AlertTok}[1]{\textcolor[rgb]{0.94,0.16,0.16}{#1}}
\newcommand{\AnnotationTok}[1]{\textcolor[rgb]{0.56,0.35,0.01}{\textbf{\textit{#1}}}}
\newcommand{\AttributeTok}[1]{\textcolor[rgb]{0.77,0.63,0.00}{#1}}
\newcommand{\BaseNTok}[1]{\textcolor[rgb]{0.00,0.00,0.81}{#1}}
\newcommand{\BuiltInTok}[1]{#1}
\newcommand{\CharTok}[1]{\textcolor[rgb]{0.31,0.60,0.02}{#1}}
\newcommand{\CommentTok}[1]{\textcolor[rgb]{0.56,0.35,0.01}{\textit{#1}}}
\newcommand{\CommentVarTok}[1]{\textcolor[rgb]{0.56,0.35,0.01}{\textbf{\textit{#1}}}}
\newcommand{\ConstantTok}[1]{\textcolor[rgb]{0.00,0.00,0.00}{#1}}
\newcommand{\ControlFlowTok}[1]{\textcolor[rgb]{0.13,0.29,0.53}{\textbf{#1}}}
\newcommand{\DataTypeTok}[1]{\textcolor[rgb]{0.13,0.29,0.53}{#1}}
\newcommand{\DecValTok}[1]{\textcolor[rgb]{0.00,0.00,0.81}{#1}}
\newcommand{\DocumentationTok}[1]{\textcolor[rgb]{0.56,0.35,0.01}{\textbf{\textit{#1}}}}
\newcommand{\ErrorTok}[1]{\textcolor[rgb]{0.64,0.00,0.00}{\textbf{#1}}}
\newcommand{\ExtensionTok}[1]{#1}
\newcommand{\FloatTok}[1]{\textcolor[rgb]{0.00,0.00,0.81}{#1}}
\newcommand{\FunctionTok}[1]{\textcolor[rgb]{0.00,0.00,0.00}{#1}}
\newcommand{\ImportTok}[1]{#1}
\newcommand{\InformationTok}[1]{\textcolor[rgb]{0.56,0.35,0.01}{\textbf{\textit{#1}}}}
\newcommand{\KeywordTok}[1]{\textcolor[rgb]{0.13,0.29,0.53}{\textbf{#1}}}
\newcommand{\NormalTok}[1]{#1}
\newcommand{\OperatorTok}[1]{\textcolor[rgb]{0.81,0.36,0.00}{\textbf{#1}}}
\newcommand{\OtherTok}[1]{\textcolor[rgb]{0.56,0.35,0.01}{#1}}
\newcommand{\PreprocessorTok}[1]{\textcolor[rgb]{0.56,0.35,0.01}{\textit{#1}}}
\newcommand{\RegionMarkerTok}[1]{#1}
\newcommand{\SpecialCharTok}[1]{\textcolor[rgb]{0.00,0.00,0.00}{#1}}
\newcommand{\SpecialStringTok}[1]{\textcolor[rgb]{0.31,0.60,0.02}{#1}}
\newcommand{\StringTok}[1]{\textcolor[rgb]{0.31,0.60,0.02}{#1}}
\newcommand{\VariableTok}[1]{\textcolor[rgb]{0.00,0.00,0.00}{#1}}
\newcommand{\VerbatimStringTok}[1]{\textcolor[rgb]{0.31,0.60,0.02}{#1}}
\newcommand{\WarningTok}[1]{\textcolor[rgb]{0.56,0.35,0.01}{\textbf{\textit{#1}}}}
\usepackage{graphicx,grffile}
\makeatletter
\def\maxwidth{\ifdim\Gin@nat@width>\linewidth\linewidth\else\Gin@nat@width\fi}
\def\maxheight{\ifdim\Gin@nat@height>\textheight\textheight\else\Gin@nat@height\fi}
\makeatother
% Scale images if necessary, so that they will not overflow the page
% margins by default, and it is still possible to overwrite the defaults
% using explicit options in \includegraphics[width, height, ...]{}
\setkeys{Gin}{width=\maxwidth,height=\maxheight,keepaspectratio}
% Set default figure placement to htbp
\makeatletter
\def\fps@figure{htbp}
\makeatother
\setlength{\emergencystretch}{3em} % prevent overfull lines
\providecommand{\tightlist}{%
  \setlength{\itemsep}{0pt}\setlength{\parskip}{0pt}}
\setcounter{secnumdepth}{-\maxdimen} % remove section numbering

\title{Metody odkrywania wiedzy}
\author{Rafał Galczak, Wojciech Gruszka}
\date{}

\begin{document}
\maketitle

\begin{Shaded}
\begin{Highlighting}[]
\KeywordTok{library}\NormalTok{(ggplot2)}
\end{Highlighting}
\end{Shaded}

\#Projekt

\hypertarget{zagadnienie}{%
\subsection{Zagadnienie}\label{zagadnienie}}

\textbf{\emph{Temat:}} \emph{Lokalna regresja (predykcja wartości
funkcji docelowej dla przykładu za pomocą modelu jednorazowego użytku
budowanego na podstawie jego „najbliższych sąsiadów''). Eksperymenty z
kilkoma różnymialgorytmami do budowania modeli lokalnych. Porównanie z
algorytmami regresji dostępnymi w R.}

W ramach projektu należało napisać z użyciem pakietu R
biblioteki/funkcji, która pozwoli na zastosowanie algorytmów regresji
dostępnych w pakiecie R w trybie ``lokalnym''. sprowadza się to do
znalezienia w zbiorze danych (który musi być do tego odpowiednio
przygotowany) zbioru trenującego dla danego przykładu z użyciem
algorytmu k-najbliższych sąsiadów. Następnie dla takiego zbioru należy
zastosować wybrany algorytm regresji z pakietu R. Jako algorytmy
budowania modeli należy zastosować dostępne w pakiecie R algorytmy
regresjii. Wykorzystane algorytmy zostały opisane w dalszej części.

W ramach projektu miały zostać przeprowadzone testy sprawdzające
działanie usyskanego kodu dla kilku algorytmów oraz zbiorów danych.

\hypertarget{czemu-warto-zajux105ux107-siux119-takim-zagadnieniem}{%
\subsection{Czemu warto zająć się takim
zagadnieniem}\label{czemu-warto-zajux105ux107-siux119-takim-zagadnieniem}}

Samo zadanie jest dość ciekawe ze względu na to, że istnieją sensowne
algorytmy pozwalające na realizację regresji lokalnej takie jak
\emph{LOES} czy \emph{LOWESS}. Jednym z powodów może być tutaj chęć
przetestowania sprawności algorytmów regresji w trybie lokalnym i
weryfikacji poprawności ich działania. Zastosowanie algorytmów
budowanaia modeli regresji w trybie lokalnym pozwala na znaczne
zmniejszenie czasu i zasobów potrzebnych do wykonania predykcji, co
wynika ze znacznego ograniczenia liczebności danych trenujących do tylko
tych z bezpośredniego obszaru (k-najbliższych sąsiadów).

\hypertarget{zbiory-danych}{%
\subsection{Zbiory danych}\label{zbiory-danych}}

Do projektu musieliśmy wybrać zbiory danych testowych które powinny
spełniać wymagania: - liczba atrybutów powinna być odpowiednio duża
(\textgreater5 atrybutów) - zmienna objaśniana powinna być ciągła lub
zachowywać sens w przypadku interpolacji (np. ocena filmu w skali
dyskretnej 1-10 zachowuje sens dla wartości 9.81) - zbiór danych
powinien zawierać odpowiednio dużo przykładów

Po długich poszukiwaniach sensownych i odpowiednio ciekawych danych do
analizy na dostępnych portalach udostępniających zbiory danych (Kaggle,
Data.World) zdecydowaliśmy się na następujące propozycje:

\hypertarget{zbiuxf3r-dancyh-wypoux17cyceux144-roweruxf3w-w-systemie-wypoux17cyczalni}{%
\subsubsection{Zbiór dancyh wypożyceń rowerów w systemie
wypożyczalni}\label{zbiuxf3r-dancyh-wypoux17cyceux144-roweruxf3w-w-systemie-wypoux17cyczalni}}

Dane pochodzą ze strony:
\href{}{https://data.world/data-society/capital-bikeshare-2011-2012}.

Zbiór zawiera 17379 przykładów dotyczących wypożyczeń rowerów w systemie
wypożyczalni w Waszyngtonie. Dostępne są atrybuty związane z czasem
wypożyczenia (dzień tygodnia, godzina), pogodą (temperatura,
wiilgotność, prędkość wiatru) oraz liczby wypożyczonych rowerów
(użytkownicy zarejsetrowani, użytkownicy niezarejestrowani, użytkownicy
łącznie).W szczególności regresję można badać w zakresie łącznej liczby
wypożyczeń. Zbiór wydaje się dość ciekawy do zbadania oraz jest
wystarczająco dużo.

\begin{Shaded}
\begin{Highlighting}[]
\NormalTok{bikes <-}\StringTok{ }\KeywordTok{read.csv}\NormalTok{(}\StringTok{"https://query.data.world/s/jcpgkqdal2ztirvoahx5dmopkpzr7q"}\NormalTok{, }\DataTypeTok{header=}\OtherTok{TRUE}\NormalTok{, }\DataTypeTok{stringsAsFactors=}\OtherTok{FALSE}\NormalTok{)}
\end{Highlighting}
\end{Shaded}

\begin{Shaded}
\begin{Highlighting}[]
\KeywordTok{max}\NormalTok{(bikes}\OperatorTok{$}\NormalTok{Season)}
\end{Highlighting}
\end{Shaded}

\begin{verbatim}
## [1] 4
\end{verbatim}

\hypertarget{zbiuxf3r-danych-uux17cywanych-samochoduxf3w}{%
\subsubsection{Zbiór danych używanych
samochodów}\label{zbiuxf3r-danych-uux17cywanych-samochoduxf3w}}

Dane pochodzą ze strony:
\href{}{https://www.kaggle.com/c/usedcarvaluation}

\begin{Shaded}
\begin{Highlighting}[]
\NormalTok{cars <-}\StringTok{ }\KeywordTok{read.csv}\NormalTok{(}\StringTok{"cars.csv"}\NormalTok{, }\DataTypeTok{header=}\OtherTok{TRUE}\NormalTok{, }\DataTypeTok{stringsAsFactors=}\OtherTok{FALSE}\NormalTok{)}
\end{Highlighting}
\end{Shaded}

\hypertarget{zbiuxf3r-danych-nieruchomoux15bci}{%
\subsubsection{Zbiór danych
nieruchomości}\label{zbiuxf3r-danych-nieruchomoux15bci}}

Dane pochodzą ze strony:
\href{}{https://www.kaggle.com/quantbruce/real-estate-price-prediction}

\begin{Shaded}
\begin{Highlighting}[]
\NormalTok{real_estate <-}\StringTok{ }\KeywordTok{read.csv}\NormalTok{(}\StringTok{"real_estate.csv"}\NormalTok{, }\DataTypeTok{header=}\OtherTok{TRUE}\NormalTok{, }\DataTypeTok{stringsAsFactors=}\OtherTok{FALSE}\NormalTok{)}
\end{Highlighting}
\end{Shaded}

Dane zostały wybrane ze względu na ciekaw

\hypertarget{implementacja}{%
\subsection{Implementacja}\label{implementacja}}

\hypertarget{normalizacja-i-denormalizacja-danych}{%
\subsubsection{Normalizacja i denormalizacja
danych}\label{normalizacja-i-denormalizacja-danych}}

W celu poprawnego zastosowania metody kNN (k najbliższych sąsiadów) do
ogreślenia zbioru terningowego dane muszą być uprzednio znormalizowane.
Jest to wymagane, ponieważ algorytm kNN bazuje na funkcji odległości.
Brak normalizacji sprawiłby, że jeden z atrybutów byłby wyróżniony ze
względu na większy zakres wartosci. Stosuje się trzy rozdzaje
normalizacji: - Liniowe przeskalowanie wartości do przedizału \([0-1]\)
dla atrybutów ograniczonych (o ustalonych wartościach minimalnej i
maksymalnej) - Przeskalowanie z użyciem rozkładu normalnego o średniej
\(Ex = 0\) i wariancji \(\sigma^{2} = 1\). - dla wartości
nieograniczonych - Przydzielenie wartości \([0-1]\) rozłożonych
równomiernie dla atrybutów dyskretnych

W pakiecie taka normalizacja może zostać wykonana automatycznie przez
zastosowanie funckcji ``.

\hypertarget{funkcja-knn-k-najbliux17cszych-sux105siaduxf3w}{%
\subsubsection{Funkcja kNN (k najbliższych
sąsiadów)}\label{funkcja-knn-k-najbliux17cszych-sux105siaduxf3w}}

Funkcja \texttt{getKnn} pozwala na wybranie spośród wskazanego zbioru

\hypertarget{model-regresji-liniowej}{%
\subsubsection{Model regresji liniowej}\label{model-regresji-liniowej}}

\hypertarget{model-drzew-regresji}{%
\subsubsection{Model drzew regresji}\label{model-drzew-regresji}}

\hypertarget{uux17cycie-pakietu}{%
\subsection{Użycie pakietu}\label{uux17cycie-pakietu}}

\hypertarget{metody-regresji-lokalnej-z-pakietu-r}{%
\subsection{Metody regresji lokalnej z pakietu
R}\label{metody-regresji-lokalnej-z-pakietu-r}}

\hypertarget{loess}{%
\subsubsection{LOESS}\label{loess}}

\hypertarget{lowess}{%
\subsubsection{LOWESS}\label{lowess}}

\hypertarget{wyniki-i-poruxf3wnanie-z-metodami-regresji-lokalnej-dostux119pnymi-w-pakiecie-r}{%
\subsection{Wyniki i porównanie z metodami regresji lokalnej dostępnymi
w pakiecie
R}\label{wyniki-i-poruxf3wnanie-z-metodami-regresji-lokalnej-dostux119pnymi-w-pakiecie-r}}

\end{document}
